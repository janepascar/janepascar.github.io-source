% Dr. Geoff Boeing - Curriculum Vitae
% Copyright 2019 Geoff Boeing
% Email: boeing@usc.edu
% Web: https://geoffboeing.com/
% Repo: https://github.com/gboeing/cv

\documentclass[12pt,letterpaper]{report}

\usepackage[T1]{fontenc} % output T1 font encoding (8-bit) so accented characters are a single glyph
\usepackage[utf8]{inputenc} % allow input of utf-8 encoded characters
\usepackage[strict,autostyle]{csquotes} % smart and nestable quote marks
\usepackage[USenglish]{babel} % automatically regionalize hyphens, quote marks, etc
\usepackage{microtype} % improves text appearance with kerning, etc
\usepackage{datetime} % enable formatting of date output
\usepackage{tabto} % make nice tabbing
\usepackage{hyperref} % enable hyperlinks and pdf metadata
\usepackage{geometry} % manually set page margins
\usepackage{enumitem} % enumerate with [resume] option
\usepackage{titlesec} % allow custom section fonts

% what is your name?
\newcommand{\myname}{Jane A. Pascar}

% define a default font face and set it as the body font
\usepackage{crimson} % document's serif font
\usepackage{helvet}  % document's sans serif font

% define how far to tab for list items with left-aligned date - different font faces need different widths
\newcommand{\listtabwidth}{1.75cm}

% set name font to title the document
\newcommand{\namefont}[1]{{\normalfont\bfseries\Huge{#1}}}

% set section heading fonts and before/after spacing
\SetTracking{encoding=*}{20}
\titleformat{\section}{\sffamily\small\bfseries\lsstyle\uppercase}{}{}{}{}
\titlespacing{\section}{0pt}{24pt plus 4pt minus 2pt}{12pt plus 2pt minus 2pt}

% set subsection heading fonts and before/after spacing
\titleformat{\subsection}{\sffamily\footnotesize\bfseries}{}{}{}{}
\titlespacing{\subsection}{0pt}{12pt plus 4pt minus 2pt}{8pt plus 2pt minus 2pt}

% set page margins
\geometry{body={6.5in, 9.0in},
    left=1.0in,
    top=1.0in}

% prevent paragraph indentation
\setlength\parindent{0em}

% define space between list items
\newcommand{\listitemspace}{0.15em}

% make unordered lists without bullets and use compact spacing
\renewenvironment{itemize}
{\begin{list}{}{\setlength{\leftmargin}{0em}
            \setlength{\parskip}{0em}
            \setlength{\itemsep}{\listitemspace}
            \setlength{\parsep}{\listitemspace}}}
    {\end{list}}

% make tabbed lists so content is left-aligned next to years
\TabPositions{\listtabwidth}
\newlist{tablist}{description}{3}
\setlist[tablist]{leftmargin=\listtabwidth,
    labelindent=0em,
    topsep=0em,
    partopsep=0em,
    itemsep=\listitemspace,
    parsep=\listitemspace,
    font=\normalfont}

% print the month and year only when using \today
\newdateformat{monthyeardate}{\monthname[\THEMONTH] \THEYEAR}

% define hyperlink appearance and metadata for pdf properties
\hypersetup{
    colorlinks  = true,
    urlcolor    = black,
    pdfauthor   = {\myname},
    pdfkeywords = {},
    pdftitle    = {\myname: Curriculum Vitae},
    pdfsubject  = {Curriculum Vitae},
    pdfpagemode = UseNone
}

\begin{document}
    \raggedright

    % display name as the document title
    \namefont{\myname}

    % contact info
    \vspace{1em}
    \begin{minipage}[t]{0.68\textwidth}
        Center for Reproductive Evolution \\
        Department of Biological Sciences \\
        Syracuse University
        \\Syracuse, New York
    \end{minipage}
    \begin{minipage}[t]{0.31\textwidth}
        Email: \href{mailto:japascar@syr.edu}{japascar@syr.edu} \\
        Web: \href{http://janepascar.github.io}{janepascar.github.io} \\
        Twitter: \href{https://twitter.com/janepascar}{@janepascar} \\
        Github: \href{https://github.com/janepascar}{@janepascar}
    \end{minipage}
    \vspace{0.5em}
    
    
    
    \section*{Research Interests}

    \begin{itemize}

        \item I study the tripartite relationship between mosquito genetics, microbiome composition and Plasmodium, the protozoan parasite responsible for causing malaria. I am most interested in understanding and quantifying systemic differences in the microbiome between infected and uninfected individuals. To address these questions I use a combination of computational biology, biogeography, and microbial ecology.


    \end{itemize}


    \section*{Education}

    \begin{tablist}

        \item[2018--] \tab Ph.D. in Biological Sciences \\
        				   Syracuse University, Center for Reproductive Evolution
        
        \item[2017] \tab B.Sc. in Zoology,  \textit{Cum laude} with University Honors \\
        			     State University of New York at Oswego

    \end{tablist}

    \section*{Research Experience}

    \begin{tablist}
    
    	\item[2018--] \tab Graduate Research, Syracuse University, Center for Reproductive Evolution \\
    					   Advisor: Steve Dorus \\
    					   - Quantifying natural microbiome variation in the presence of pathogenic parasites%:

        \item[2017--18] \tab Research Technician, University of Southern California, Marine and Environmental Biology \\
                            PI: Suzanne Edmands \\
                            - Hybrid incompatibilities and the evolution of mito-nuclear conflicts%:

        \item[2016--17] \tab Undergraduate Honors Thesis Research, SUNY Oswego, Department of Biological Sciences \\
                            Advisor: Chris Chandler \\
                            - Taxonomically diverse survey of {{\emph{Wolbachia}}} presence and genomic analyses%:
		
		\item[2016] \tab Visiting Scholar, Université de Poitiers, Laboratory of Ecological and Biological Interactions \\
                            PI: Dr. Richard Cordaux \\
                            - Surgical sex-reversal of terrestrial isopods and sex chromosome evolution%: Both female (ZW/ZZ) and male (XY/XX) heterogametic systems exist within terrestrial isopod species. Furthermore, the sex chromosomes are thought to be relatively young and homomorphic so classical cytogenetic analyses are difficult. By surgically implanting a male-specific androgenic gland into a juvenile female, a neo-male is produced (genetically female/phenotypically male). By crossing these neo-males with their sisters (same sex chromosome genotype) and analyzing the progeny sex-ratio, the heterogametic system can be inferred.

    \end{tablist}


    \section*{Publications}

    \subsection*{Peer-Reviewed Articles}

    \begin{itemize}
    	
    	\item[1.] \textbf{Pascar JA} and Chandler CH. 2018. A bioinformatics approach to identifying {{\emph{Wolbachia}}} infections in arthropods. \textit{PeerJ} 6:e5486. \href{https://doi.org/10.7717/peerj.5486}{doi:10.7717/peerj.5486}
    	
    \end{itemize}

    %\subsection*{Submitted}
    
    %\begin{itemize}
    	
    %	\item[XX.] Russell A, Borrelli S, Fontana R, Laricchiuta J, \textbf{Pascar JA}, Becking T, Giraud I, Cordaux R, and Chandler CH. 20XX. Evolutionary transition to XY sex chromosomes associated with Y-linked duplication of a male hormone gene in a terrestrial isopod. Submitted to \textit{Genome Biology and Evolution}
    	
    %\end{itemize}

    \section*{Conference Presentations}

    \subsection*{Talks}

    \begin{tablist}

        \item[2020] \tab Russell A, Borrelli S, Fontana R, Laricchiuta J, \textbf{Pascar JA}, Becking T, Giraud I, Cordaux R, Chandler CH*. A transition to XY sex chromosomes associated with Y-linked duplication of a male hormone gene in a terrestrial isopod. Society for Integrative and Comparative Biology (SICB), Austin, TX
        
        %\item[2019] \tab \textbf{Pascar JA*} and Dorus S. Characterizing microbiome complexity in the African mosquito {{\emph{Anopheles gambiae.}}} Departmental Seminar, Syracuse, NY
        
        \item[2017] \tab \textbf{Pascar JA*} and Chandler CH. Identification of novel {{\emph{Wolbachia}}} infections in arthropods using publicly accessible next-generation sequencing data. Quest, Oswego, NY \textit{• Awarded Sigma Xi/ORSP Research Award}

    \end{tablist}

    \subsection*{Posters}

    \begin{tablist}
        
        \item[2019] \tab Middleton H†*, \textbf{Pascar JA}, Dorus S. Microbiome patterns in {{\emph{Aedes aegypti}}} mosquitoes transinfected with {{\emph{Wolbachia}}}. Syracuse University Summer Undergraduate Research Conference, Syracuse, NY
        
        %\item[2019] \tab \textbf{Pascar JA*} and Dorus S. Characterizing microbiome complexity in the malaria vector {{\emph{Anopheles gambiae.}}} Departmental Seminar, Syracuse, NY
        
        \item[2018] \tab \textbf{Pascar JA*}, Watson ET, Edmands S. Properties of sex-biased gene expression in the absence of sex chromosomes. Population, Evolution and Quantitative Genetics (PEQG), Madison, WI
        
        \item[2017] \tab \textbf{Pascar JA*} and Chandler CH. Testing for prevalence of {{\emph{Wolbachia}}} in the terrestrial isopod species {{\emph{Porcellio laevis}}} and {{\emph{Trachelipus rathkei}}}. Society for Integrative and Comparative Biology (SICB), New Orleans, LA
        
        \item[2017] \tab \textbf{Pascar JA*} and Chandler CH. Testing for prevalence of {{\emph{Wolbachia}}} in the terrestrial isopod species {{\emph{Porcellio laevis}}} and {{\emph{Trachelipus rathkei}}}. RISE Scholarly \& Creative Activities Symposium, Oswego, NY
        
        \item * indicates presenter
        \item † indicates undergraduate advisee

        
    \end{tablist}

   
    \section*{Grants and Awards}

    \subsection*{Awards and Honors}

    \begin{tablist}

        \item[2017] \tab Sigma Xi/SUNY Oswego Office of Research and Sponsored Programs Research Award

    \end{tablist}

    \subsection*{Grants and Fellowships}

    \begin{tablist}
    	\item Total Awards Received = \$295,500 (Only proposals awarded are listed)

        \item[2019--24] \tab National Science Foundation Graduate Research Fellowship (\$138,000)
        
        \item[2019] \tab Syracuse University Dept. of Biology Travel Grant (\$500)
        
        \item[2019] \tab Syracuse University Graduate Student Organization Travel Grant (\$500)
        
        \item[2018--19] \tab Syracuse University STEM Research Fellowship (\$136,500)
        
        \item[2013--17] \tab SUNY Oswego Presidential Scholarship (\$20,000)
        
    \end{tablist}



    \section*{Teaching Experience}

    \subsection*{SUNY Oswego}
    
    \begin{itemize}
    	
    	\item Microbiology Lab (BIO 310)
    	\item Molecular and Cellular Biology (BIO 120)
    	\item Introduction to the Honors Program (HON 150)
    
	\end{itemize}
    	

    \section*{Service}
    
    \subsection*{Mentoring}

    \begin{itemize}
    	
    	\item	Henry Middleton - Cornell University, Computer Science
        
    \end{itemize}

    \subsection*{Outreach}

    \begin{tablist}
    
    	\item[2019--] \tab \textbf{Letters to a Pre-Scientist}: A program designed to pair middle school students from low-income schools with scientists at all career levels. Over the course of an academic year I exchanged letters with student discussing what college/graduate school, overcoming obstacles, and careers.
        
        \item[2019] \tab \textbf{Frontiers in Science}: Co-led a lab activity for high school students on Drosophila genetics. Students studied flies under dissecting microscopes and identified the various visible mutations. 
        
        \item[2018] \tab \textbf{USC McMorrow Neighborhood Academic Initiative}: The NAI is an intensive pre-college enrichment program for low-income students in 6th through 12th grade in the Los Angeles area. I traveled with a group of these student to the Wrigley Marine Science Center on Catalina Island to collect intertidal copepods and conduct short-term observational studies. This introduced the students to evolutionary topics such as the Mother’s curse hypothesis and maintenance of polygenic sex determination. Throughout the year I helped host lab tours for participating teachers and their students.
        
        \item[2018] \tab \textbf{Darwin Day Workshop}: I helped facilitate a workshop with ~50 local K-12 teachers on lab activities they can implement in their classrooms to supplement their teaching of evolution.

    \end{tablist}


    \section*{Media Coverage}

    \begin{tablist}
    	
    	\item[2019] \tab {{\emph{Syracuse University News.}}} Students Earn 2019 National Science Foundation Awards. Apr 24.
    	
    	\item[2018]	\tab {{\emph{Wrigley Institute Research Blog.}}} Sex-biased expression in the copepod {{\emph{Tigriopus.}}} May 16.
        
    \end{tablist}

    % display today's date as Month Year after a vertical space below the end of the text
    \begin{center}
        %\vspace{6em}
        \vfill
        Updated \monthyeardate\today
    \end{center}


\end{document}